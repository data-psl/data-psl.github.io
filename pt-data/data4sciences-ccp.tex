% !TeX encoding = UTF-8
% !TeX spellcheck = fr_FR
\documentclass[a4paper,portrait,fontscale=0.8]{baposter}
\usepackage[french]{babel}
\usepackage[utf8]{inputenc}
%\usepackage{arev}
\usepackage[T1]{fontenc}

\usepackage{enumitem,xcolor,url}

% Color Definition
\definecolor{psl}{rgb}{0.16, 0.32, 0.75}
\definecolor{standardfontcolor}{RGB}{0,0,0}  % black
\definecolor{bordercol}{RGB}{113,113,113}    
\definecolor{headercol1}{RGB}{255,255,255}   % white 
\definecolor{headercol2}{RGB}{113,113,113}
\definecolor{headerfontcol}{RGB}{0,0,0}
\definecolor{boxcolor}{RGB}{255,255,255}

\newlist{citemize}{itemize}{1}
\setlist[citemize]{label=\textcolor{psl}{\textbullet}}

\newcommand{\data}{{\color{psl}\sc Data@PSL }}

\begin{document}
	\typeout{Poster rendering started}
	
% \background{
% 	\begin{tikzpicture}[remember qpicture,overlay]%
% 	\draw (current page.north west)+(-2em,2em) node[anchor=north west]
% 	{\includegraphics[height=1.1\textheight]{figures/background}};
% 	\end{tikzpicture}
% }

\color{standardfontcolor}

\begin{poster}{
    grid=false,
    columns=2,
    % colspacing=length
    headerheight=0.170\textheight,
    eyecatcher=false, 
    borderColor=psl,
    headerColorOne=headercol1,
    headerColorTwo=headercol1,
    headerFontColor=psl,
    % Only simple background color used, no shading, so boxColorTwo isn't necessary
    boxColorOne=boxcolor,
    headershape=rounded,
    headerfont=\bfseries\Large,
    textborder=rounded,
    headerborder=open,
    boxshade=plain,
    background=none
  }
	%%% Eye Cacther %%%%%%%%%%%%%%%%%%%%%%%%%%%%%%%%%%%%%%%%%%%%%%%%%%%%%%%%%%%%%%%
	{
		Eye Catcher, empty if option eyecatcher=false - unused
	}
%----------------------------------------------------------------------------------------
%	TITLE AND AUTHOR NAME
%----------------------------------------------------------------------------------------
%
{ 
	 % \textsf Sans Serif
  { {\color{psl} Cross-cutting\\[1ex] program on AI \\[1ex] for the Sciences}\\[1ex]
  }
}
{%
  \color{standardfontcolor}    {\bf \data program}\\\vspace{2ex}
}
% University/lab logo
{\includegraphics[height=2cm]{LOGO-PSL.png}} 
%
%

\headerbox{Overview}{name=overview,column=0,row=0.05,span=2}{
{ \large \begin{citemize}
  \item A unique opportunity to add skills in machine learning and data science.
  \item In a close interaction with your major track, from the master to PhD.
  \item Enjoy the diversity of PSL. 
  \item Open new perspectives in your scientific fields of interest.
  \item Acquire new methods, new tools to address new challenges.
  \end{citemize}
}
~
}




\headerbox{How to}{name=howto,column=0,row=0.35,span=1}{
  Kick-off:
  \begin{citemize}
  \item What can be taken from  your present and past curriculum  in the ECTS  ?
  \item Follow the preparatory weeks of \data to upgrade your skills in data science.
  \end{citemize}
  Then:
  \begin{citemize}
  \item Follow intensive weeks of \data to learn how AI can contribute to other sciences.
  \item Do an internship involving data science (a single but extended report).
  \item Join up a Hackathon and follow a "Ethics of AI" course.
  \end{citemize}
  \vspace{0.25ex}
}
 
% \headerbox{The specific courses}{name=specifics,column=1,below=overview,span=1}{
\headerbox{The specific courses}{name=specifics,column=1,row=0.35,span=1}{
  The \data program organizes specific courses to build your minor:\\

  \noindent{\color{psl}\bf Preparatory weeks} allow you to acquire the necessary background in math and data science. Mandatory to  follow the rest of the minor (end of August/early September).  \\

  \noindent{\color{psl}\bf Intensive weeks} gather students  with different profiles to tackle a scientific challenge (PSL weeks calendar). For example in genomics, finance, physics, and more.   \\

  \noindent{\color{psl}\bf Hackathon: } within a  team, you will practice on real datasets for 10 weeks, with a weekly supervision.   
}

\headerbox{Validation of 30 ECTS}{name=features,column=0,row=0.815,span=1}{
  \begin{citemize}
  \item 2 preparatory weeks (6 ECTS)
  \item An intensive week (3 ECTS)
  \item AI and Ethics course (3 ECTS)
  \item Hackathon, Internship (9 to 12 ECTS)
  \item Additional courses of your curriculum, approved by the \data program
  \end{citemize}
  
}
\headerbox{Contact}{name=features,column=1,row=0.815,span=1}{
  \begin{citemize}
  \item Contact the direction of your master  
  \item The official web site: \\[-0.75cm]
  \end{citemize}
  \begin{flushright}
    \url{https://bit.ly/39tfHNT}
  \end{flushright}
  \begin{citemize}
  \item  Breaking news and contact:\\[-0.75cm]
  \end{citemize}
  \begin{flushright}
    \url{data-psl.github.io}
  \end{flushright}
  \vspace{0.04cm}
}
\end{poster}
\end{document}
